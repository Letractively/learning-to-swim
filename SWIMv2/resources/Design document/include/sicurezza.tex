\section{Sicurezza}

Per accedere a tutte le funzioni e per motivi di sicurezza sono richiesti la registrazione ed il login: entrambe le procedure sono molto semplici, in quanto l'interfaccia della web app è progettata per essere il più
user-friendly possibile. Tutte le informazioni che gli utenti forniscono nella fase di registrazione sono memorizzate in un database di tipo relazionale MySql.

L'ultima sezione del Design Document riguarda le metodologie impiegate per garantire la sicurezza del sistema: vengono impiegati SHA1 per crittografare la password prima di inserirla nel database e HTTPS per garantire connessioni protette.

\section{Precisazioni sul Design Document}
\begin{itemize}
\item [$\textcolor{red}{\star}$]Coerentemente con le rettifiche presenti nel RASD le parti riguardanti le conversazioni pubbliche non sono state implementate e quindi descrivono un modello completo di come potrebbe essere SWIM in una versione successiva.

\item[$\textcolor{red}{\star}$]Alcune chiavi delle entità del DBMS sono state modificate per ragioni implementative.
\end{itemize}