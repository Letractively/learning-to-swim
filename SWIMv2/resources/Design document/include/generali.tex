\section{Generali}

\subsection{Breve Panoramica del sistema}
Il sistema SWIMv2 è pensato per tutte quelle persone che hanno bisogno di un professionista in un determinato ambito e desiderano cercarlo online. Per accedere a tutte le funzionalità
e per motivi di sicurezza sono richiesti la registrazione ed il login: entrambe le procedure sono molto semplici, in quanto l'interfaccia della web app è progettata per essere il più
user-friendly possibile. Tutte le informazioni che gli utenti forniscono nella fase di registrazione sono memorizzate in un database di tipo relazionale (mysql) e, per ragioni di
praticità e sicurezza non sono visibili in chiaro, bensì sotto forma di hash salato (vedi la sezione ``Sicurezza''); in particolare la password è crittografata con crittografia
!!! TODO : DA DEFINIRSI !!!

\subsection{Scopo di questo documento}
Lo scopo di questo documento è quello di descrivere nel dettaglio, anche tramite l'ausilio di diagrammi E-R, UML, UX, ecc. , lo scheletro del sistema, i suoi vari componenti (database,
interfaccia utente e logica applicativa), le tecnologie utilizzate e la gestione della sicurezza, il tutto opportunamente spiegato anche minuziosamente ove è richiesto per una
ottimale comprensione di SWIMv2.

\subsection{Assunzioni e rettifiche}
Nello scrivere questo documento abbiamo deciso di precisare alcune nostre assunzioni rispetto al sistema e alla progettazione:
\begin{itemize}
 \item Gli amministratori del sistema sono e saranno sempre i tre autori di questo articolo.
 \item Il set di abilità iniziali predefinito di sistema è costituito da:
 \begin{itemize}
  \item Ingegnere del Software
  \item Master dei videogiochi ``Final Fantasy''
  \item Esperto di Computer Security
  \item Esperto di Crittografia
  \item Esperto di calcio mondiale
  \item Esperto di Probabilità e Statistica
  \item Esperto della società Mozilla
  \item Esperto di automobili
  \item Esperto di Apple
  \item Esperto di Linux e dell'open source
 \end{itemize}
 \item Un ospite non può chiedere aiuto da nessuna parte: può solo effettuare ricerche tra gli utenti registrati.
 \item Conseguenza del punto precedente è che non esiste una bacheca pubblica come inizialmente da noi ipotizzato.
\end{itemize}


\subsection{Glossario}
In questo documento vengono usate le seguenti sigle/abbreviazioni:
\begin{itemize}
 \item HW: Hardware
 \item SW: Software
 \item DD: Design Document
 \item PP: Project Planning
 \item RASD: Requirements Analysis and Specification Document
 \item DB: Database
 \item J2EE: Java Enterprise Edition
 \item UX: User Experience (diagram)
 \item DBMS: DataBase Management System
 \item AS: Application Server
\end{itemize}
Sono tutte sigle abbastanza note, ma, a scanso di equivoci è sempre meglio precisare (a cominciare dal fatto che questa sezione non è un filler, ma ha una sua utilità).

\pagebreak

\section{Software impiegato per il documento e per il sistema}
Per scrivere questo documento sono stati impiegati i seguenti software:
\begin{itemize}
 \item Kile, un editor per scrivere in \LaTeX, e compilare il sorgente in un file PDF
 \item LiveTex, una distribuzione di \TeX , senza cui Kile non può compilare
 \item JDER, un editor di diagrammi E-R
 \item Mysql Workbench per il diagramma di vista logica del database
 \item !!!TODO : ciò che ho usato per lo UX !!!
\end{itemize}
Invece per lo sviluppo di SWIMv2 sono stati impiegati i seguenti software e le seguenti tecnologie:
\begin{itemize}
 \item Eclipse Juno per J2EE per la scrittura del codice java/jsp
 \item Openjdk 1.7 come versione di java, senza cui Eclipse non può nemmeno partire
 \item Mysql Server come server per il DB
 \item JBoss AS 5 come server locale su cui far girare il sistema
 \item WaveMaker !!!O chi per lui !!! per disegnare l'interfaccia grafica
\end{itemize}

