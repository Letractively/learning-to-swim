\section{Generali}


\subsection{Panoramica della piattaforma}
La piattaforma sviluppata ha il compito di promuovere, tramite l’utilizzo di una web application, la collaborazione e la condivisione di informazioni utili alla risoluzione di problemi tra gli utenti.\\[1\baselineskip]Nel dettaglio, il presente documento di Design definisce la struttura concettuale e funzionale fornendo una precisa descrizione delle guidelines che saranno seguite nello sviluppo e nel deployment dell’applicazione.\\[1\baselineskip]Il documento è inoltre conforme con le specifiche presenti nel RASD.\\[1\baselineskip]Il sistema è utilizzato da tre categorie di utenti: gli ospiti, gli utenti registrati e gli amministratori. Tramite un sistema di autenticazione il sistema controlla chi può svolgere determinate azioni in base al proprio ruolo.\\[1\baselineskip]Per evitare l’accesso a servizi non consentiti si è deciso di utilizzare il Role-Based Access Control. Con tale sistema di autenticazione si consentirà di usufruire solamente dei servizi specifici per il proprio ruolo.


\subsection{Scopo di questo documento}
Lo scopo di questo documento è quello di descrivere nel dettaglio, anche tramite l'ausilio di diagrammi E-R, UML, UX, ecc. , lo scheletro del sistema, i suoi vari componenti (database,
interfaccia utente e logica applicativa), le tecnologie utilizzate e la gestione della sicurezza per comprendere al meglio come è stata sviluppata SWIMv2.

\subsection{Glossario}
In questo documento vengono usate le seguenti sigle/abbreviazioni:
\begin{itemize}
 \item HW: Hardware
 \item SW: Software
 \item DD: Design Document
 \item PP: Project Planning
 \item RASD: Requirements Analysis and Specification Document
 \item DB: Database
 \item J2EE: Java Enterprise Edition
 \item UX: User Experience (diagram)
 \item DBMS: DataBase Management System
 \item AS: Application Server
\end{itemize}

\pagebreak


